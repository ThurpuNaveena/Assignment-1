\documentclass[journal,12pt,twocolumn]{IEEEtran}

\usepackage{setspace}
\usepackage{gensymb} 

\singlespacing


\usepackage[cmex10]{amsmath}

\usepackage{amsthm}


\usepackage{mathrsfs}
\usepackage{txfonts}
\usepackage{stfloats}
\usepackage{bm}
\usepackage{cite}
\usepackage{cases}
\usepackage{subfig}

\usepackage{longtable}
\usepackage{multirow}

\usepackage{enumitem}
\usepackage{mathtools}
\usepackage{steinmetz}
\usepackage{tikz}
\usepackage{circuitikz}
\usepackage{verbatim}
\usepackage{tfrupee}
\usepackage[breaklinks=true]{hyperref}
\usepackage{graphicx}
\usepackage{tkz-euclide}

\usetikzlibrary{calc,math}
\usepackage{listings}
    \usepackage{color}                                            %%
    \usepackage{array}                                            %%
    \usepackage{longtable}                                        %%
    \usepackage{calc}                                             %%
    \usepackage{multirow}                                         %%
    \usepackage{hhline}                                           %%
    \usepackage{ifthen}                                           %%
    \usepackage{lscape}     
\usepackage{multicol}
\usepackage{chngcntr}

\DeclareMathOperator*{\Res}{Res}

\renewcommand\thesection{\arabic{section}}
\renewcommand\thesubsection{\thesection.\arabic{subsection}}
\renewcommand\thesubsubsection{\thesubsection.\arabic{subsection}}

\renewcommand\thesectiondis{\arabic{section}}
\renewcommand\thesubsectiondis{\thesectiondis.\arabic{subsection}}
\renewcommand\thesubsubsectiondis{\thesubsectiondis.\arabic{subsection}}


\hyphenation{op-tical net-works semi-conduc-tor}
\def\inputGnumericTable{}                                 %%

\lstset{
%language=C,
frame=single, 
breaklines=true,
columns=fullflexible
}
\begin{document}


\newtheorem{theorem}{Theorem}[section]
\newtheorem{problem}{Problem}
\newtheorem{proposition}{Proposition}[section]
\newtheorem{lemma}{Lemma}[section]
\newtheorem{corollary}[theorem]{Corollary}
\newtheorem{example}{Example}[section]
\newtheorem{definition}[problem]{Definition}

\newcommand{\BEQA}{\begin{eqnarray}}
\newcommand{\EEQA}{\end{eqnarray}}
\newcommand{\define}{\stackrel{\triangle}{=}}
\bibliographystyle{IEEEtran}
\providecommand{\mbf}{\mathbf}
\providecommand{\pr}[1]{\ensuremath{\Pr\left(#1\right)}}
\providecommand{\qfunc}[1]{\ensuremath{Q\left(#1\right)}}
\providecommand{\sbrak}[1]{\ensuremath{{}\left[#1\right]}}
\providecommand{\lsbrak}[1]{\ensuremath{{}\left[#1\right.}}
\providecommand{\rsbrak}[1]{\ensuremath{{}\left.#1\right]}}
\providecommand{\brak}[1]{\ensuremath{\left(#1\right)}}
\providecommand{\lbrak}[1]{\ensuremath{\left(#1\right.}}
\providecommand{\rbrak}[1]{\ensuremath{\left.#1\right)}}
\providecommand{\cbrak}[1]{\ensuremath{\left\{#1\right\}}}
\providecommand{\lcbrak}[1]{\ensuremath{\left\{#1\right.}}
\providecommand{\rcbrak}[1]{\ensuremath{\left.#1\right\}}}
\theoremstyle{remark}
\newtheorem{rem}{Remark}
\newcommand{\sgn}{\mathop{\mathrm{sgn}}}
\providecommand{\abs}[1]{\left\vert#1\right\vert}
\providecommand{\res}[1]{\Res\displaylimits_{#1}} 
\providecommand{\norm}[1]{\left\lVert#1\right\rVert}
%\providecommand{\norm}[1]{\lVert#1\rVert}
\providecommand{\mtx}[1]{\mathbf{#1}}
\providecommand{\mean}[1]{E\left[ #1 \right]}
\providecommand{\fourier}{\overset{\mathcal{F}}{ \rightleftharpoons}}
%\providecommand{\hilbert}{\overset{\mathcal{H}}{ \rightleftharpoons}}
\providecommand{\system}{\overset{\mathcal{H}}{ \longleftrightarrow}}
	%\newcommand{\solution}[2]{\textbf{Solution:}{#1}}
\newcommand{\solution}{\noindent \textbf{Solution: }}
\newcommand{\cosec}{\,\text{cosec}\,}
\providecommand{\dec}[2]{\ensuremath{\overset{#1}{\underset{#2}{\gtrless}}}}
\newcommand{\myvec}[1]{\ensuremath{\begin{pmatrix}#1\end{pmatrix}}}
\newcommand{\mydet}[1]{\ensuremath{\begin{vmatrix}#1\end{vmatrix}}}
\numberwithin{equation}{subsection}
\makeatletter
\@addtoreset{figure}{problem}
\makeatother
\let\StandardTheFigure\thefigure
\let\vec\mathbf
\renewcommand{\thefigure}{\theproblem}
\def\putbox#1#2#3{\makebox[0in][l]{\makebox[#1][l]{}\raisebox{\baselineskip}[0in][0in]{\raisebox{#2}[0in][0in]{#3}}}}
     \def\rightbox#1{\makebox[0in][r]{#1}}
     \def\centbox#1{\makebox[0in]{#1}}
     \def\topbox#1{\raisebox{-\baselineskip}[0in][0in]{#1}}
     \def\midbox#1{\raisebox{-0.5\baselineskip}[0in][0in]{#1}}
\vspace{3cm}
\title{ASSIGNMENT-1}
\author{T.Naveena}
\maketitle
\newpage
\bigskip
\renewcommand{\thefigure}{\theenumi}
\renewcommand{\thetable}{\theenumi}
Download all python codes from 
\begin{lstlisting}
https://github.com/ThurpuNaveena/Assignment-1/blob/main/ASSIGNMENT1/assignment1.py
\end{lstlisting}
%
and latex-tikz codes from 
\begin{lstlisting}
https://github.com/ThurpuNaveena/Assignment-1/blob/main/ASSIGNMENT1/main.tex
\end{lstlisting}
%
\section{QUESTION NO-2.2}
Construct an isosceles triangle whose base is $a$ = 8cm and altitude $AD$= $h$=4cm. 
%
\section{SOLUTION}
 Given,
\begin{align}
\ a = 8,
\ h = 4
\end{align}
we use the Pythagoras theorem,
\begin{align}
\ c^2 &=a^2+b^2
\\
\implies c^2 &=4^2+4^2 
\\
\implies c^2 &=32 
\\
\implies c &=5.6
\end{align}
 
\begin{align}
\vec{b} = \vec{c}
\end{align}
Two sides are equal so $\triangle ABC$ is isosceles triangle
 Let the vertices of  $\triangle ABC$ and $\vec{D}$ be 
\begin{align}
\label{eq:tri_basic}
\vec{A} = \myvec{p\\q}, \vec{B} = \myvec{0\\0}, \vec{C} = \myvec{a\\0}, \vec{D} = \myvec{p\\0}
\end{align}
%

Then
\begin{align}
\label{eq:c_tricoord}
AB &= \norm{\vec{A}-\vec{B}}^2 = \norm{\vec{A}}^2  = c^2 \quad \because \vec{B} = \vec{0}
\\
\label{eq:a_tricoord}
BC &= \norm{\vec{C}-\vec{B}}^2 = \norm{\vec{C}}^2  = a^2
\\
AC &= \norm{\vec{A}-\vec{C}}^2 =    b^2
\label{eq:b_tricoord}
\end{align}
%
From \eqref{eq:b_tricoord},
\begin{align}
b^2 &=\norm{\vec{A}-\vec{C}}^2 = \norm{\vec{A}-\vec{C}}^T\norm{\vec{A}-\vec{C}}  
\\
&= \vec{A}^T\vec{A}+\vec{C}^T\vec{C}-\vec{A}^T\vec{C} - \vec{C}^T\vec{A} 
\\
&= \norm{\vec{A}}^2 + \norm{\vec{C}}^2 - 2\vec{A}^T\vec{C} \quad \brak{\because \vec{A}^T\vec{C} = \vec{C}^T\vec{A} } 
\label{eq:tri_const_norm_ac}
\\
&= a^2+c^2-2ap
\end{align}
%
yielding
\begin{align}
p&= \frac{a^2+c^2-b^2}{2a}
\\
p&= \frac{8^2+(5.6)^2-(5.6)^2}{16}
\\
p&= 4
\end{align}
%
From \eqref{eq:c_tricoord}, 
\begin{align}
\norm{\vec{A}}^2 &= c^2 = p^2+q^2
\\
\implies q&= \pm \sqrt{c^2-p^2}
\\
\implies q&= \pm \sqrt{(5.6)^2-4^2}
\\
\implies q&= \pm 3.9
\end{align}
 Let the vertices of  $\triangle ABC$ and $\vec{D}$ be 
\begin{align}
\label{eq:tri_basic}
\vec{A} = \myvec{4\\3.9}, \vec{B} = \myvec{0\\0}, \vec{C} = \myvec{8\\0}, \vec{D} = \myvec{4\\0}
\end{align}
\numberwithin{figure}{section}
\begin{figure}[!ht]
\centering
\includegraphics[width=\columnwidth]{diagram.png}
\caption{isosceles triangle$\triangle ABC$}
\label{fig:tri_SSS_triangle}	
\end{figure}
\end{document}
